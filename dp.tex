\newif\ifdblatexpdf \ifx\pdfoutput\undefined \else \ifx\pdfoutput\relax \else\ifnum\pdfoutput>0 \dblatexpdftrue \fi \fi \fi
% ------------------------------------------------------------
% Autogenerated LaTeX file for books
% db2latex RELEASE: 0.8pre1
% db2latex VERSION: $Id: VERSION.xml,v 1.6 2004/01/31 12:47:11 j-devenish Exp $
% fithesis VERSION: 1.40
% ------------------------------------------------------------
\def\clsclass{rapport3}
\documentclass{fithesis}
% --------------------------------------------
% MetaFont and MetaPost logo support
% --------------------------------------------
\usepackage{mflogo}
% --------------------------------------------
% Load fithesis param 
% --------------------------------------------
\thesistitle{Realtime communication between web browsers}
\thesissubtitle{Master thesis}
\thesisstudent{Pavel Smolka}
\thesiswoman{false}
\thesislang{en}
\thesisyear{2013}
\thesisfaculty{fi}
\thesisadvisor{doc.
      RNDr. Tom� Pitner, Ph.D.}
% --------------------------------------------
\label{idp283328}% --------------------------------------------
% Load graphicx package with pdf if needed 
% --------------------------------------------
\ifdblatexpdf
\usepackage[pdftex]{graphicx}
\pdfcompresslevel=9
\else
\usepackage{graphicx}
\fi
\usepackage{anysize}
\marginsize{3cm}{2.5cm}{3.5cm}{3.5cm}

\makeatletter
% redefine the listoffigures and listoftables so that the name of the chapter
% is printed whenever there are figures or tables from that chapter. encourage
% pagebreak prior to the name of the chapter (discourage orphans).
\let\save@@chapter\@chapter
\let\save@@l@figure\l@figure
\let\the@l@figure@leader\relax
\def\@chapter[#1]#2{\save@@chapter[{#1}]{#2}%
\addtocontents{lof}{\protect\def\the@l@figure@leader{\protect\pagebreak[0]\protect\contentsline{chapter}{\protect\numberline{\thechapter}#1}{}{\thepage}}}%
\addtocontents{lot}{\protect\def\the@l@figure@leader{\protect\pagebreak[0]\protect\contentsline{chapter}{\protect\numberline{\thechapter}#1}{}{\thepage}}}%
}
\renewcommand*\l@figure{\the@l@figure@leader\let\the@l@figure@leader\relax\save@@l@figure}
\let\l@table\l@figure
\makeatother
\usepackage{fancyhdr}
\renewcommand{\headrulewidth}{0.4pt}
\renewcommand{\footrulewidth}{0.4pt}
% Safeguard against long headers.
\IfFileExists{truncate.sty}{
\usepackage{truncate}
% Use an ellipsis when text would be larger than x% of the text width.
% Preserve left/right text alignment using \hfill (works for English).
\fancyhead[ol]{\truncate{0.49\textwidth}{\sl\leftmark}}
\fancyhead[er]{\truncate{0.49\textwidth}{\hfill\sl\rightmark}}
\fancyhead[el]{\truncate{0.49\textwidth}{\sl\leftmark}}
\fancyhead[or]{\truncate{0.49\textwidth}{\hfill\sl\rightmark}}
}{\typeout{WARNING: truncate.sty wasn't available and functionality was skipped.}}
\pagestyle{fancy}
% ---------------------- 
% Most Common Packages   
% ---------------------- 
\usepackage{latexsym}         
\usepackage{enumerate}         
\usepackage{fancybox}      
\usepackage{float}       
\usepackage{ragged2e}       
\usepackage{fancyvrb}         
\makeatletter\@namedef{FV@fontfamily@default}{\def\FV@FontScanPrep{}\def\FV@FontFamily{}}\makeatother
\fvset{obeytabs=true,tabsize=3}
\makeatletter
\let\dblatex@center\center\let\dblatex@endcenter\endcenter
\def\dblatex@nolistI{\leftmargin\leftmargini\topsep\z@ \parsep\parskip \itemsep\z@}
\def\center{\let\@listi\dblatex@nolistI\@listi\dblatex@center\let\@listi\@listI\@listi}
\def\endcenter{\dblatex@endcenter}
\makeatother
\usepackage{rotating}         
\usepackage{subfigure}         
\usepackage{tabularx}         
\usepackage{url}         
% --------------------------------------------
% Load hyperref package with pdf if needed 
% --------------------------------------------
\ifdblatexpdf
\usepackage[pdftex,bookmarksnumbered,colorlinks,backref,bookmarks,breaklinks,linktocpage,plainpages=false, pdfstartview=FitH, plainpages=false, pdfpagelabels, unicode]{hyperref}
\else
\usepackage[bookmarksnumbered,colorlinks,backref,bookmarks,breaklinks,linktocpage,plainpages=false, plainpages=false, pdfpagelabels]{hyperref}
\fi
% --------------------------------------------
% ----------------------------------------------
% Define a new LaTeX environment (adminipage)
% ----------------------------------------------
\newenvironment{admminipage}%
{ % this code corresponds to the \begin{adminipage} command
 \begin{Sbox}%
 \begin{minipage}%
} %done
{ % this code corresponds to the \end{adminipage} command
 \end{minipage}
 \end{Sbox}
 \fbox{\TheSbox}
} %done
% ----------------------------------------------
% Define a new LaTeX length (admlength)
% ----------------------------------------------
\newlength{\admlength}
% ----------------------------------------------
% Define a new LaTeX environment (admonition)
% With 2 parameters:
% #1 The file (e.g. note.pdf)
% #2 The caption
% ----------------------------------------------
\newenvironment{admonition}[2] 
{ % this code corresponds to the \begin{admonition} command
 \hspace{0mm}\newline\hspace*\fill\newline
 \noindent
 \setlength{\fboxsep}{5pt}
 \setlength{\admlength}{\linewidth}
 \addtolength{\admlength}{-10\fboxsep}
 \addtolength{\admlength}{-10\fboxrule}
 \admminipage{\admlength}
 {\bfseries \sc\large{#2}} \newline
 \\[1mm]
 \sffamily
 \includegraphics[width=1cm]{#1}
 \addtolength{\admlength}{-1cm}
 \addtolength{\admlength}{-20pt}
 \begin{minipage}[lt]{\admlength}
 \parskip=0.5\baselineskip \advance\parskip by 0pt plus 2pt
} %done
{ % this code corresponds to the \end{admonition} command
 \vspace{5mm} 
 \end{minipage}
 \endadmminipage
 \vspace{.5em}
 \par
}
% --------------------------------------------
% Commands to manage/style/create floats      
% figures, tables, algorithms, examples, eqn  
% --------------------------------------------
 \floatstyle{plain}
 \restylefloat{figure}
 \floatstyle{plain}
 \restylefloat{table}
 \floatstyle{plain}
 \newfloat{program}{ht}{lop}[section]
 \floatstyle{plain}
 \newfloat{example}{ht}{loe}[section]
 \floatname{example}{Example}
 \floatstyle{plain}
 \newfloat{dbequation}{ht}{loe}[section]
 \floatname{dbequation}{Equation}
 \floatstyle{boxed}
 \newfloat{algorithm}{ht}{loa}[section]
 \floatname{algorithm}{Algorithm}
\ifdblatexpdf
\DeclareGraphicsExtensions{.pdf,.png,.jpg}
\else
\DeclareGraphicsExtensions{.eps}
\fi
% --------------------------------------------
% $latex.caption.swapskip enabled for $formal.title.placement support
\newlength{\docbooktolatextempskip}
\newcommand{\captionswapskip}{\setlength{\docbooktolatextempskip}{\abovecaptionskip}\setlength{\abovecaptionskip}{\belowcaptionskip}\setlength{\belowcaptionskip}{\docbooktolatextempskip}}
% Guard against a problem with old package versions.
\makeatletter
\AtBeginDocument{
\DeclareRobustCommand\ref{\@refstar}
\DeclareRobustCommand\pageref{\@pagerefstar}
}
\makeatother
% --------------------------------------------
\makeatletter
\newcommand{\dbz}{\penalty \z@}
\newcommand{\docbooktolatexpipe}{\ensuremath{|}\dbz}
\newskip\docbooktolatexoldparskip
\newcommand{\docbooktolatexnoparskip}{\docbooktolatexoldparskip=\parskip\parskip=0pt plus 1pt}
\newcommand{\docbooktolatexrestoreparskip}{\parskip=\docbooktolatexoldparskip}
\def\cleardoublepage{\clearpage\if@twoside \ifodd\c@page\else\hbox{}\thispagestyle{empty}\newpage\if@twocolumn\hbox{}\newpage\fi\fi\fi}
\usepackage[latin2]{inputenc}
\usepackage[T1]{fontenc}

\ifx\dblatex@chaptersmark\@undefined\def\dblatex@chaptersmark#1{\markboth{\MakeUppercase{#1}}{}}\fi
\let\save@makeschapterhead\@makeschapterhead
\def\dblatex@makeschapterhead#1{\vspace*{-80pt}\save@makeschapterhead{#1}}
\def\@makeschapterhead#1{\dblatex@makeschapterhead{#1}\dblatex@chaptersmark{#1}}

\AtBeginDocument{\ifx\refname\@undefined\let\docbooktolatexbibname\bibname\def\docbooktolatexbibnamex{\bibname}\else\let\docbooktolatexbibname\refname\def\docbooktolatexbibnamex{\refname}\fi}
% Facilitate use of \cite with \label
\newcommand{\docbooktolatexbibaux}[2]{%
  \protected@write\@auxout{}{\string\global\string\@namedef{docbooktolatexcite@#1}{#2}}
}
% Provide support for bibliography `subsection' environments with titles
\newenvironment{docbooktolatexbibliography}[3]{
   \begingroup
   \let\save@@chapter\chapter
   \let\save@@section\section
   \let\save@@@mkboth\@mkboth
   \let\save@@bibname\bibname
   \let\save@@refname\refname
   \let\@mkboth\@gobbletwo
   \def\@tempa{#3}
   \def\@tempb{}
   \ifx\@tempa\@tempb
      \let\chapter\@gobbletwo
      \let\section\@gobbletwo
      \let\bibname\relax
   \else
      \let\chapter#2
      \let\section#2
      \let\bibname\@tempa
   \fi
   \let\refname\bibname
   \begin{thebibliography}{#1}
}{
   \end{thebibliography}
   \let\chapter\save@@chapter
   \let\section\save@@section
   \let\@mkboth\save@@@mkboth
   \let\bibname\save@@bibname
   \let\refname\save@@refname
   \endgroup
}

%\usepackage{cite}
%\renewcommand\citeleft{(}  % parentheses around list
%\renewcommand\citeright{)} % parentheses around list
\newcommand{\docbooktolatexcite}[2]{%
  \@ifundefined{docbooktolatexcite@#1}%
  {\cite{#1}}%
  {\def\@docbooktolatextemp{#2}\ifx\@docbooktolatextemp\@empty%
   \cite{\@nameuse{docbooktolatexcite@#1}}%
   \else\cite[#2]{\@nameuse{docbooktolatexcite@#1}}%
   \fi%
  }%
}
\newcommand{\docbooktolatexbackcite}[1]{%
  \ifx\Hy@backout\@undefined\else%
    \@ifundefined{docbooktolatexcite@#1}{%
      % emit warning?
    }{%
      \ifBR@verbose%
        \PackageInfo{backref}{back cite \string`#1\string' as \string`\@nameuse{docbooktolatexcite@#1}\string'}%
      \fi%
      \Hy@backout{\@nameuse{docbooktolatexcite@#1}}%
    }%
  \fi%
}

% --------------------------------------------
% A way to honour <footnoteref>s
% Blame j-devenish (at) users.sourceforge.net
% In any other LaTeX context, this would probably go into a style file.
\newcommand{\docbooktolatexusefootnoteref}[1]{\@ifundefined{@fn@label@#1}%
  {\hbox{\@textsuperscript{\normalfont ?}}%
    \@latex@warning{Footnote label `#1' was not defined}}%
  {\@nameuse{@fn@label@#1}}}
\newcommand{\docbooktolatexmakefootnoteref}[1]{%
  \protected@write\@auxout{}%
    {\global\string\@namedef{@fn@label@#1}{\@makefnmark}}%
  \@namedef{@fn@label@#1}{\hbox{\@textsuperscript{\normalfont ?}}}%
  }

\makeindex
% index labeling helper
\newif\ifdocbooktolatexprintindex\docbooktolatexprintindextrue
\let\dbtolatex@@theindex\theindex
\let\dbtolatex@@endtheindex\endtheindex
\@ifundefined{@openrighttrue}{\newif\if@openright}{}
\def\theindex{\relax}
\def\endtheindex{\relax}
\newenvironment{dbtolatexindex}[2]
   {
\if@openright\cleardoublepage\else\clearpage\fi
\let\dbtolatex@@indexname\indexname
\def\dbtolatex@current@indexname{#2}
\ifx\dbtolatex@current@indexname\@empty                                                                                             \def\dbtolatex@current@indexname{\dbtolatex@@indexname}
\fi
\def\dbtolatex@indexlabel{%
 \ifnum \c@secnumdepth >\m@ne \ifx\c@chapter\undefined\refstepcounter{section}\else\refstepcounter{chapter}\fi\fi%
 \label{#1}\hypertarget{#1}{\dbtolatex@current@indexname}%
 \global\docbooktolatexprintindexfalse}
\def\indexname{\ifdocbooktolatexprintindex\dbtolatex@indexlabel\else\dbtolatex@current@indexname\fi}
\dbtolatex@@theindex
   }
   {
\dbtolatex@@endtheindex\let\indexname\dbtolatex@@indexname
   }

\newlength\saveparskip \newlength\saveparindent
\newlength\tempparskip \newlength\tempparindent

\def\docbooktolatexgobble{\expandafter\@gobble}
% Prevent multiple openings of the same aux file
% (happens when backref is used with multiple bibliography environments)
\ifx\AfterBeginDocument\undefined\let\AfterBeginDocument\AtBeginDocument\fi
\AfterBeginDocument{
   \let\latex@@starttoc\@starttoc
   \def\@starttoc#1{%
      \@ifundefined{docbooktolatex@aux#1}{%
         \global\@namedef{docbooktolatex@aux#1}{}%
         \latex@@starttoc{#1}%
      }{}
   }
}
% --------------------------------------------
% Hacks for honouring row/entry/@align
% (\hspace not effective when in paragraph mode)
% Naming convention for these macros is:
% 'docbooktolatex' 'align' {alignment-type} {position-within-entry}
% where r = right, l = left, c = centre
\newcommand{\docbooktolatex@align}[2]{\protect\ifvmode#1\else\ifx\LT@@tabarray\@undefined#2\else#1\fi\fi}
\newcommand{\docbooktolatexalignll}{\docbooktolatex@align{\raggedright}{}}
\newcommand{\docbooktolatexalignlr}{\docbooktolatex@align{}{\hspace*\fill}}
\newcommand{\docbooktolatexaligncl}{\docbooktolatex@align{\centering}{\hfill}}
\newcommand{\docbooktolatexaligncr}{\docbooktolatex@align{}{\hspace*\fill}}
\newcommand{\docbooktolatexalignrl}{\protect\ifvmode\raggedleft\else\hfill\fi}
\newcommand{\docbooktolatexalignrr}{}
\ifx\captionswapskip\@undefined\newcommand{\captionswapskip}{}\fi
\makeatother
\title{\bfseries Realtime communication between web browsers\\[12pt]\normalsize Master thesis}
\author{Pavel Smolka}
% --------------------------------------------
\makeglossary
% --------------------------------------------

\setcounter{tocdepth}{4}

\setcounter{secnumdepth}{4}
\begin{document}
% --------------------------------------------
% Useing fithesis
% --------------------------------------------
\FrontMatter
\ThesisTitlePage
\begin{ThesisDeclaration}
\DeclarationText
\AdvisorName
\end{ThesisDeclaration}

% --------------------------------------------
% Thanks 
% --------------------------------------------
\begin{ThesisThanks}
Thanks to everyone... TODO\end{ThesisThanks}

% --------------------------------------------
% Abstract 
% --------------------------------------------
\begin{ThesisAbstract}

Abstrakt, if any TODO
\end{ThesisAbstract}

% --------------------------------------------
% KeyWords 
% --------------------------------------------
\begin{ThesisKeyWords}
keyword 1, keyword 2, TODO\end{ThesisKeyWords}

\makeatletter
\def\dbtolatex@contentsid{idp4586880}
\def\dbtolatex@@contentsname{\latex@@contentsname}
\let\latex@@contentsname\contentsname
\newif\ifdocbooktolatexcontentsname\docbooktolatexcontentsnametrue
\def\dbtolatex@contentslabel{%
 \label{\dbtolatex@contentsid}\hypertarget{\dbtolatex@contentsid}{\dbtolatex@@contentsname}%
 \global\docbooktolatexcontentsnamefalse}
\def\contentsname{\ifdocbooktolatexcontentsname\dbtolatex@contentslabel\else\dbtolatex@@contentsname\fi}
\let\save@@@mkboth\@mkboth
\let\@mkboth\@gobbletwo
\tableofcontents
\let\@mkboth\save@@@mkboth
\let\contentsname\latex@@contentsname
\Hy@writebookmark{}{\dbtolatex@@contentsname}{\dbtolatex@contentsid}{0}{toc}%
\makeatother
				\MainMatter

% -------------------------------------------------------------
% Chapter Introduction 
% ------------------------------------------------------------- 	
\chapter{Introduction}
\label{uvod}\hypertarget{uvod}{}%

Thesis introduction, motivation.

The thesis should embrace the topics of text chat in web browsers, interprocess communication and the possibilities of multimedia content transfer (audio/video) - including the current options of capturing multimedia directly from the web browser.

Describe the need for browser chat application in Celebrio, also the business cases.

% -------------------------------------------------------------
% Chapter XMPP Protocol 
% ------------------------------------------------------------- 	
\chapter{XMPP Protocol}
\label{idp4589104}\hypertarget{idp4589104}{}%

At first, this chapter should describe the important parts of XMPP protocol (mostly just picking the appropriate references).

Later on, description of BOSH extension, including the advantages and limitations. Describe also the connection to HTTP protocol.

Jingle extension - multimedia. Does this deserve complete chapter???

Pub-sub extensions for interprocess communication.

% -------------------------------------------------------------
% Chapter Tools 
% ------------------------------------------------------------- 	
\chapter{Tools}
\label{idp209632}\hypertarget{idp209632}{}%

Describe the tools that can be used to implement RTC in WB (and which were used to implement Celebrio Talker)

JS + Ember.js

Strophe (simple XMPP in Javascript)

Strophe plugins

Possible server-side implementations (JAXL, XMPPHP, ...)

Google hangouts API

OpenTok

WebRTC (http://www.webrtc.org/)

% -------------------------------------------------------------
% Chapter Talker 
% ------------------------------------------------------------- 	
\chapter{Talker}
\label{idp136976}\hypertarget{idp136976}{}%

Describe the Talker application in Celebrio.

Mention what we expected from the app (value proposition)

Then, application analysis, design and implmementation.

% -------------------------------------------------------------
% Chapter Inter-process Communication Framework 
% ------------------------------------------------------------- 	
\chapter{Inter-process Communication Framework}
\label{idp138928}\hypertarget{idp138928}{}%

Implement and describe framework for IPC in Celebrio.

First, the lightweight one (which we already have), then implement the "heavy" one, if there's enough time.

% -------------------------------------------------------------
% Chapter Conclusion 
% ------------------------------------------------------------- 	
\chapter{Conclusion}
\label{idp140640}\hypertarget{idp140640}{}%

conclusion

% ------------------------   
% Section 
\section{Another part of the conclusion}
\label{idp141728}\hypertarget{idp141728}{}%

Another part of the conclusion... just to have subchapter here
% ------------------------------------------- 
%
%  Bibliography - chapter
%
% ------------------------------------------- 
\begin{thebibliography}{123}\hypertarget{idp143056}{}

% ............. biblioentry 
\bibitem{xmpp-the-definitive-guide}\docbooktolatexbibaux{idp143312}{xmpp-the-definitive-guide}
\hypertarget{idp143312}
Saint-Andre, P. and Smith, K. and Tron�on, R.: \emph{XMPP: The Definitive Guide}, Sebastopol: O'Reilly, 2009, 978-0-596-52126-4, 320. 

\end{thebibliography}
\addcontentsline{toc}{chapter}{Bibliography}
\setlength\saveparskip\parskip
\setlength\saveparindent\parindent
\begin{dbtolatexindex}{idp151216}{}
\setlength\tempparskip\parskip \setlength\tempparindent\parindent
\parskip\saveparskip \parindent\saveparindent
\noindent \indexspace
\parskip\tempparskip
\parindent\tempparindent
\makeatletter\@input@{\jobname.ind}\makeatother
\addcontentsline{toc}{chapter}{Index}
\end{dbtolatexindex}

\newcommand{\dbappendix}[1]{\chapter{#1}}%
% ------------------------------------------------------------- 
% Appendices start here
% -------------------------------------------------------------
\appendix

% -------------------------------------------------------------
% appendix:  Appendix, if any... 
% ------------------------------------------------------------- 	
\dbappendix{Appendix, if any...}
\label{my-appendix}\hypertarget{my-appendix}{}%

appendix content

\end{document}

